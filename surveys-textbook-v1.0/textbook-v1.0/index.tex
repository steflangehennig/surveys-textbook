% Options for packages loaded elsewhere
\PassOptionsToPackage{unicode}{hyperref}
\PassOptionsToPackage{hyphens}{url}
\PassOptionsToPackage{dvipsnames,svgnames,x11names}{xcolor}
%
\documentclass[
  letterpaper,
  DIV=11,
  numbers=noendperiod]{scrreprt}

\usepackage{amsmath,amssymb}
\usepackage{lmodern}
\usepackage{iftex}
\ifPDFTeX
  \usepackage[T1]{fontenc}
  \usepackage[utf8]{inputenc}
  \usepackage{textcomp} % provide euro and other symbols
\else % if luatex or xetex
  \usepackage{unicode-math}
  \defaultfontfeatures{Scale=MatchLowercase}
  \defaultfontfeatures[\rmfamily]{Ligatures=TeX,Scale=1}
\fi
% Use upquote if available, for straight quotes in verbatim environments
\IfFileExists{upquote.sty}{\usepackage{upquote}}{}
\IfFileExists{microtype.sty}{% use microtype if available
  \usepackage[]{microtype}
  \UseMicrotypeSet[protrusion]{basicmath} % disable protrusion for tt fonts
}{}
\makeatletter
\@ifundefined{KOMAClassName}{% if non-KOMA class
  \IfFileExists{parskip.sty}{%
    \usepackage{parskip}
  }{% else
    \setlength{\parindent}{0pt}
    \setlength{\parskip}{6pt plus 2pt minus 1pt}}
}{% if KOMA class
  \KOMAoptions{parskip=half}}
\makeatother
\usepackage{xcolor}
\setlength{\emergencystretch}{3em} % prevent overfull lines
\setcounter{secnumdepth}{5}
% Make \paragraph and \subparagraph free-standing
\ifx\paragraph\undefined\else
  \let\oldparagraph\paragraph
  \renewcommand{\paragraph}[1]{\oldparagraph{#1}\mbox{}}
\fi
\ifx\subparagraph\undefined\else
  \let\oldsubparagraph\subparagraph
  \renewcommand{\subparagraph}[1]{\oldsubparagraph{#1}\mbox{}}
\fi

\usepackage{color}
\usepackage{fancyvrb}
\newcommand{\VerbBar}{|}
\newcommand{\VERB}{\Verb[commandchars=\\\{\}]}
\DefineVerbatimEnvironment{Highlighting}{Verbatim}{commandchars=\\\{\}}
% Add ',fontsize=\small' for more characters per line
\usepackage{framed}
\definecolor{shadecolor}{RGB}{241,243,245}
\newenvironment{Shaded}{\begin{snugshade}}{\end{snugshade}}
\newcommand{\AlertTok}[1]{\textcolor[rgb]{0.68,0.00,0.00}{#1}}
\newcommand{\AnnotationTok}[1]{\textcolor[rgb]{0.37,0.37,0.37}{#1}}
\newcommand{\AttributeTok}[1]{\textcolor[rgb]{0.40,0.45,0.13}{#1}}
\newcommand{\BaseNTok}[1]{\textcolor[rgb]{0.68,0.00,0.00}{#1}}
\newcommand{\BuiltInTok}[1]{\textcolor[rgb]{0.00,0.23,0.31}{#1}}
\newcommand{\CharTok}[1]{\textcolor[rgb]{0.13,0.47,0.30}{#1}}
\newcommand{\CommentTok}[1]{\textcolor[rgb]{0.37,0.37,0.37}{#1}}
\newcommand{\CommentVarTok}[1]{\textcolor[rgb]{0.37,0.37,0.37}{\textit{#1}}}
\newcommand{\ConstantTok}[1]{\textcolor[rgb]{0.56,0.35,0.01}{#1}}
\newcommand{\ControlFlowTok}[1]{\textcolor[rgb]{0.00,0.23,0.31}{#1}}
\newcommand{\DataTypeTok}[1]{\textcolor[rgb]{0.68,0.00,0.00}{#1}}
\newcommand{\DecValTok}[1]{\textcolor[rgb]{0.68,0.00,0.00}{#1}}
\newcommand{\DocumentationTok}[1]{\textcolor[rgb]{0.37,0.37,0.37}{\textit{#1}}}
\newcommand{\ErrorTok}[1]{\textcolor[rgb]{0.68,0.00,0.00}{#1}}
\newcommand{\ExtensionTok}[1]{\textcolor[rgb]{0.00,0.23,0.31}{#1}}
\newcommand{\FloatTok}[1]{\textcolor[rgb]{0.68,0.00,0.00}{#1}}
\newcommand{\FunctionTok}[1]{\textcolor[rgb]{0.28,0.35,0.67}{#1}}
\newcommand{\ImportTok}[1]{\textcolor[rgb]{0.00,0.46,0.62}{#1}}
\newcommand{\InformationTok}[1]{\textcolor[rgb]{0.37,0.37,0.37}{#1}}
\newcommand{\KeywordTok}[1]{\textcolor[rgb]{0.00,0.23,0.31}{#1}}
\newcommand{\NormalTok}[1]{\textcolor[rgb]{0.00,0.23,0.31}{#1}}
\newcommand{\OperatorTok}[1]{\textcolor[rgb]{0.37,0.37,0.37}{#1}}
\newcommand{\OtherTok}[1]{\textcolor[rgb]{0.00,0.23,0.31}{#1}}
\newcommand{\PreprocessorTok}[1]{\textcolor[rgb]{0.68,0.00,0.00}{#1}}
\newcommand{\RegionMarkerTok}[1]{\textcolor[rgb]{0.00,0.23,0.31}{#1}}
\newcommand{\SpecialCharTok}[1]{\textcolor[rgb]{0.37,0.37,0.37}{#1}}
\newcommand{\SpecialStringTok}[1]{\textcolor[rgb]{0.13,0.47,0.30}{#1}}
\newcommand{\StringTok}[1]{\textcolor[rgb]{0.13,0.47,0.30}{#1}}
\newcommand{\VariableTok}[1]{\textcolor[rgb]{0.07,0.07,0.07}{#1}}
\newcommand{\VerbatimStringTok}[1]{\textcolor[rgb]{0.13,0.47,0.30}{#1}}
\newcommand{\WarningTok}[1]{\textcolor[rgb]{0.37,0.37,0.37}{\textit{#1}}}

\providecommand{\tightlist}{%
  \setlength{\itemsep}{0pt}\setlength{\parskip}{0pt}}\usepackage{longtable,booktabs,array}
\usepackage{calc} % for calculating minipage widths
% Correct order of tables after \paragraph or \subparagraph
\usepackage{etoolbox}
\makeatletter
\patchcmd\longtable{\par}{\if@noskipsec\mbox{}\fi\par}{}{}
\makeatother
% Allow footnotes in longtable head/foot
\IfFileExists{footnotehyper.sty}{\usepackage{footnotehyper}}{\usepackage{footnote}}
\makesavenoteenv{longtable}
\usepackage{graphicx}
\makeatletter
\def\maxwidth{\ifdim\Gin@nat@width>\linewidth\linewidth\else\Gin@nat@width\fi}
\def\maxheight{\ifdim\Gin@nat@height>\textheight\textheight\else\Gin@nat@height\fi}
\makeatother
% Scale images if necessary, so that they will not overflow the page
% margins by default, and it is still possible to overwrite the defaults
% using explicit options in \includegraphics[width, height, ...]{}
\setkeys{Gin}{width=\maxwidth,height=\maxheight,keepaspectratio}
% Set default figure placement to htbp
\makeatletter
\def\fps@figure{htbp}
\makeatother
\newlength{\cslhangindent}
\setlength{\cslhangindent}{1.5em}
\newlength{\csllabelwidth}
\setlength{\csllabelwidth}{3em}
\newlength{\cslentryspacingunit} % times entry-spacing
\setlength{\cslentryspacingunit}{\parskip}
\newenvironment{CSLReferences}[2] % #1 hanging-ident, #2 entry spacing
 {% don't indent paragraphs
  \setlength{\parindent}{0pt}
  % turn on hanging indent if param 1 is 1
  \ifodd #1
  \let\oldpar\par
  \def\par{\hangindent=\cslhangindent\oldpar}
  \fi
  % set entry spacing
  \setlength{\parskip}{#2\cslentryspacingunit}
 }%
 {}
\usepackage{calc}
\newcommand{\CSLBlock}[1]{#1\hfill\break}
\newcommand{\CSLLeftMargin}[1]{\parbox[t]{\csllabelwidth}{#1}}
\newcommand{\CSLRightInline}[1]{\parbox[t]{\linewidth - \csllabelwidth}{#1}\break}
\newcommand{\CSLIndent}[1]{\hspace{\cslhangindent}#1}

\KOMAoption{captions}{tableheading}
\makeatletter
\makeatother
\makeatletter
\@ifpackageloaded{bookmark}{}{\usepackage{bookmark}}
\makeatother
\makeatletter
\@ifpackageloaded{caption}{}{\usepackage{caption}}
\AtBeginDocument{%
\ifdefined\contentsname
  \renewcommand*\contentsname{Table of contents}
\else
  \newcommand\contentsname{Table of contents}
\fi
\ifdefined\listfigurename
  \renewcommand*\listfigurename{List of Figures}
\else
  \newcommand\listfigurename{List of Figures}
\fi
\ifdefined\listtablename
  \renewcommand*\listtablename{List of Tables}
\else
  \newcommand\listtablename{List of Tables}
\fi
\ifdefined\figurename
  \renewcommand*\figurename{Figure}
\else
  \newcommand\figurename{Figure}
\fi
\ifdefined\tablename
  \renewcommand*\tablename{Table}
\else
  \newcommand\tablename{Table}
\fi
}
\@ifpackageloaded{float}{}{\usepackage{float}}
\floatstyle{ruled}
\@ifundefined{c@chapter}{\newfloat{codelisting}{h}{lop}}{\newfloat{codelisting}{h}{lop}[chapter]}
\floatname{codelisting}{Listing}
\newcommand*\listoflistings{\listof{codelisting}{List of Listings}}
\makeatother
\makeatletter
\@ifpackageloaded{caption}{}{\usepackage{caption}}
\@ifpackageloaded{subcaption}{}{\usepackage{subcaption}}
\makeatother
\makeatletter
\@ifpackageloaded{tcolorbox}{}{\usepackage[many]{tcolorbox}}
\makeatother
\makeatletter
\@ifundefined{shadecolor}{\definecolor{shadecolor}{rgb}{.97, .97, .97}}
\makeatother
\makeatletter
\makeatother
\ifLuaTeX
  \usepackage{selnolig}  % disable illegal ligatures
\fi
\IfFileExists{bookmark.sty}{\usepackage{bookmark}}{\usepackage{hyperref}}
\IfFileExists{xurl.sty}{\usepackage{xurl}}{} % add URL line breaks if available
\urlstyle{same} % disable monospaced font for URLs
\hypersetup{
  pdftitle={Survey Methodologies for Social Scientists},
  pdfauthor={Carey E. Stapleton \& Stefani Langehennig},
  colorlinks=true,
  linkcolor={blue},
  filecolor={Maroon},
  citecolor={Blue},
  urlcolor={Blue},
  pdfcreator={LaTeX via pandoc}}

\title{Survey Methodologies for Social Scientists}
\author{Carey E. Stapleton \& Stefani Langehennig}
\date{10/10/2022}

\begin{document}
\maketitle
\ifdefined\Shaded\renewenvironment{Shaded}{\begin{tcolorbox}[frame hidden, interior hidden, sharp corners, enhanced, boxrule=0pt, breakable, borderline west={3pt}{0pt}{shadecolor}]}{\end{tcolorbox}}\fi

\renewcommand*\contentsname{Table of contents}
{
\hypersetup{linkcolor=}
\setcounter{tocdepth}{2}
\tableofcontents
}
\bookmarksetup{startatroot}

\hypertarget{welcome}{%
\chapter*{Welcome}\label{welcome}}
\addcontentsline{toc}{chapter}{Welcome}

This is the github repository for the work-in-progress first edition of
\textbf{Survey Methodologies for Social Scientists}.

The goal of this book is to teach you how to do create, conduct, and
analyze surveys with R.

\hypertarget{acknowledgements}{%
\section*{Acknowledgements}\label{acknowledgements}}
\addcontentsline{toc}{section}{Acknowledgements}

More to come!

\bookmarksetup{startatroot}

\hypertarget{ch01intro}{%
\chapter{Introduction}\label{ch01intro}}

This book is geared toward undergraduates learning survey methodologies.
This includes, but is not limited to, the psychology of the survey
response process; creating and disseminating survey questions; survey
modes; sampling and weighting; ethics; and survey analysis and
reporting.

\hypertarget{learning-objectives}{%
\section{Learning Objectives}\label{learning-objectives}}

\bookmarksetup{startatroot}

\hypertarget{psychology-of-survey-response-response}{%
\chapter{Psychology of Survey Response
Response}\label{psychology-of-survey-response-response}}

Outlines the psychology processes that occur when a participant responds
to a survey.

\hypertarget{learning-objectives-1}{%
\section{Learning Objectives}\label{learning-objectives-1}}

\bookmarksetup{startatroot}

\hypertarget{survey-ethics}{%
\chapter{Survey Ethics}\label{survey-ethics}}

\hypertarget{learning-objectives-2}{%
\section{Learning Objectives}\label{learning-objectives-2}}

\bookmarksetup{startatroot}

\hypertarget{questionnaire-design}{%
\chapter{Questionnaire Design}\label{questionnaire-design}}

\hypertarget{learning-objectives-3}{%
\section{Learning Objectives}\label{learning-objectives-3}}

\bookmarksetup{startatroot}

\hypertarget{survey-modes}{%
\chapter{Survey Modes}\label{survey-modes}}

\hypertarget{learning-objectives-4}{%
\section{Learning Objectives}\label{learning-objectives-4}}

\bookmarksetup{startatroot}

\hypertarget{survey-sampling}{%
\chapter{Survey Sampling}\label{survey-sampling}}

What is the sample you need for the research questions you have?

\begin{itemize}
\tightlist
\item
  Nationally representative
\item
  Convenience
\item
  Online platforms
\end{itemize}

\hypertarget{learning-objectives-5}{%
\section{Learning Objectives}\label{learning-objectives-5}}

\hypertarget{probability-sampling}{%
\section{Probability sampling}\label{probability-sampling}}

\hypertarget{non-probability-sampling}{%
\section{Non-probability sampling}\label{non-probability-sampling}}

\bookmarksetup{startatroot}

\hypertarget{survey-weighting}{%
\chapter{Survey Weighting}\label{survey-weighting}}

\hypertarget{step-by-step-guide-to-creating-basic-rake-weights-in-r}{%
\section{Step-By-Step Guide to Creating Basic Rake Weights in
R}\label{step-by-step-guide-to-creating-basic-rake-weights-in-r}}

\textbf{Note: This tutorial uses the \texttt{anesrake} package to
calculate the survey weights. They are many other packages to calculate
weights so this is just one possible approach that could successfully be
used to create survey weights.}

Survey weights are widely used in survey research for a variety of
purposes. In this tutorial, we will be focusing on one specific form of
survey weights called a ``rake weight''. Rake weights are typically used
to make the survey sample match the target population on a set of
demographic, and sometimes attitudinal, measures. They are used to
ensure the sample's demographics match the target population's
demographics. This numerical correction will change how much each
individual case in your dataset is contributing to the overall, or
sub-group, mean values across your sample data.

First, we load necessary packages to compute and analyze the weights. If
a package is not installed on your machine, you must first install it
before this chunk of code will run.

\hypertarget{import-your-survey-data-into-r}{%
\section{Import your survey data into
R}\label{import-your-survey-data-into-r}}

We need to import our survey data into R. The way we do this will vary
by the format of your data, but in this case the data is saved as a
``.dta'' file so we will use the \texttt{haven} package to import it.

You should always examine your data and the base R function
\texttt{head} shows the first 5 cases along with all of your column
labels.

\begin{verbatim}
 [1] "caseid"         "pid_4"          "ideo5"          "gov_choice"    
 [5] "prop_111"       "prop_112"       "trump_app"      "hick_app"      
 [9] "gardner_app"    "cong_app"       "scotus_app"     "pot_law"       
[13] "gambling"       "fracking"       "gun_control"    "anger"         
[17] "pride"          "hope"           "disgust"        "worry"         
[21] "trump_app2"     "hick_app2"      "gardner_app2"   "cong_app2"     
[25] "scotus_app2"    "pot_law2"       "gambling2"      "fracking2"     
[29] "gun_control2"   "old_weight_old" "pid_x"          "sex"           
[33] "race_4"         "speakspanish"   "marstat"        "child18"       
[37] "employ"         "faminc_new"     "casscd"         "religiosity"   
[41] "age_group"      "educ"          
\end{verbatim}

\hypertarget{save-your-target-population-demographic-parameters}{%
\section{Save Your Target Population Demographic
Parameters}\label{save-your-target-population-demographic-parameters}}

You will need to know the target population proportion for each of the
variables you wish to weight your sample data on. How easy it will be to
find your population values will be based on your specific target
population.

Some populations will be relatively easy to find (e.g.~think adult
demographic proportions in the United States from the Census, CPS, or
ACS results and all the sub-geographic levels that accompany them), but
others won't be as easy. Sometimes, you cannot know your target
population proportions so in those cases you will not be able to weight
your survey sample data.

In this chunk of R code, we are creating the target population
parameters for two specific demographic variables measured in our sample
political poll data. This was a political poll conducted in October 2018
with the sample consisting of likely Colorado voters in the then
upcoming 2018 election. This gubernatorial election year poll measured
multiple things including: - 2018 Colorado Gubernatorial Preference -
Jared Polis or Walker Stapleton - Policy Questions: Marijuana
Legalization, Fracking, Gun Control laws - Approval ratings: President
Donald Trump, Governor John Hickenlooper (at the time), US Congress -
Demographic questions for survey weighting purposes

\hypertarget{saving-new-vectors-with-target-population-demographic-values}{%
\subsection{Saving New Vectors With Target Population Demographic
Values}\label{saving-new-vectors-with-target-population-demographic-values}}

Using this data, let's create some survey weights. To illustrate the
principle, we will start with basic weights using just two demographic
variables commonly used in calculating survey weights: sex
(unfortunately only biological sex was collected in this survey) and age
(split into 5 categories). We must save a vector of data with the target
population demographic proportions, so in this case we will save two
vectors one called \texttt{sex} and one called \texttt{age\_group}.

There are two critical things to get correct in this step.

\textbf{1. Matching Names} The names we give these vectors matter and
must match the names of the appropriate demographic variable in your
sample data. Since the vector names we chose were sex and age\_group,
the variable names in the sample data must be exactly sex and
age\_group. Otherwise, the code will not be able to match the two and
will fail.

\textbf{2. Matching Orders} The second critical thing to get correct is
the order the proportion values are entered into the vector must match
the order the proportion values are stored in the sample data. In this
example, the order of proportions stored in the sex variable in the
sample data is (female, male) so the values we give the sex vector must
be in that exact order as well. The same is true for the age\_group
variable, which has 5 groups in the sample data: 18-29, 30-39, 40-49,
50-64, 65+. The proportion order in our vector for the age\_group must
match that exactly as well otherwise you are creating incorrect weights
or best-case scenario getting an error message.

\begin{Shaded}
\begin{Highlighting}[]
\NormalTok{sex }\OtherTok{\textless{}{-}} \FunctionTok{c}\NormalTok{(.}\DecValTok{525}\NormalTok{, .}\DecValTok{475}\NormalTok{)  }\CommentTok{\#Target values for females and males; label order (female, male)}
\FunctionTok{sum}\NormalTok{(sex) }\CommentTok{\#proportions should = 1 so this checks that it does}
\end{Highlighting}
\end{Shaded}

\begin{verbatim}
[1] 1
\end{verbatim}

\begin{Shaded}
\begin{Highlighting}[]
\NormalTok{age\_group  }\OtherTok{\textless{}{-}} \FunctionTok{c}\NormalTok{(.}\DecValTok{182}\NormalTok{, .}\DecValTok{203}\NormalTok{, .}\DecValTok{17}\NormalTok{, .}\DecValTok{218}\NormalTok{, .}\DecValTok{227}\NormalTok{)  }\CommentTok{\#Target values for 5 age groups; 18{-}29, 30{-}39, 40{-}49, 50{-}64, 65+}
\FunctionTok{sum}\NormalTok{(age\_group) }\CommentTok{\#proportions should = 1 so this checks that it does}
\end{Highlighting}
\end{Shaded}

\begin{verbatim}
[1] 1
\end{verbatim}

First, let's look at the unweighted values in both the age and sex
variables.

\begin{Shaded}
\begin{Highlighting}[]
\CommentTok{\#Shows the unweighted size of the age\_group demographics}
\NormalTok{sample }\SpecialCharTok{\%\textgreater{}\%}
  \FunctionTok{group\_by}\NormalTok{(age\_group) }\SpecialCharTok{\%\textgreater{}\%}
  \FunctionTok{summarise}\NormalTok{(}\AttributeTok{n =} \FunctionTok{n}\NormalTok{()) }\SpecialCharTok{\%\textgreater{}\%} 
  \FunctionTok{mutate}\NormalTok{(}\AttributeTok{proportion =}\NormalTok{ n }\SpecialCharTok{/}\FunctionTok{sum}\NormalTok{(n))}
\end{Highlighting}
\end{Shaded}

\begin{verbatim}
# A tibble: 5 x 3
  age_group     n proportion
  <dbl+lbl> <int>      <dbl>
1 1 [18-29]    33     0.0412
2 2 [30-39]   101     0.126 
3 3 [40-49]   118     0.148 
4 4 [50-64]   306     0.382 
5 5 [65+]     242     0.302 
\end{verbatim}

\begin{Shaded}
\begin{Highlighting}[]
\CommentTok{\#Shows the unweighted size of the sex demographics}
\NormalTok{sample }\SpecialCharTok{\%\textgreater{}\%}
  \FunctionTok{group\_by}\NormalTok{(sex) }\SpecialCharTok{\%\textgreater{}\%}
  \FunctionTok{summarise}\NormalTok{(}\AttributeTok{n =} \FunctionTok{n}\NormalTok{()) }\SpecialCharTok{\%\textgreater{}\%} 
  \FunctionTok{mutate}\NormalTok{(}\AttributeTok{proportion =}\NormalTok{ n }\SpecialCharTok{/}\FunctionTok{sum}\NormalTok{(n))}
\end{Highlighting}
\end{Shaded}

\begin{verbatim}
# A tibble: 2 x 3
  sex            n proportion
  <dbl+lbl>  <int>      <dbl>
1 1 [Female]   423      0.529
2 2 [Male]     377      0.471
\end{verbatim}

\hypertarget{calculating-your-rake-weights}{%
\section{Calculating Your Rake
Weights}\label{calculating-your-rake-weights}}

Now that we have our target population parameters saved in vectors for
each demographic variable we plan to weight our survey data on and our
sample data with matching names and orders, we can begin to create our
survey weights.

\textbf{1. Create List} First, we create a list that merges the two
demographic vectors for use in the raking process. Remember, this names
must match the column names in the sample data. We give this list the
name of \texttt{targets} to reflect this is the target population
parameters we want to match the sample data to. We then give the column
names to match with the sample data.

\textbf{2. Calculate the Weights} Now, it is time to create some survey
weights using the \texttt{anesrake} function. This function has many
possible items that could be used, with all the possible items listed in
the following R chunk. You should view the R documentation for all
possible things it can do.

For our purposes, we will be focusing on a few things that will be
noted. We will calculate a new dataframe called \texttt{myweights} where
we input the \textsubscript{targets} list, the name of our sample data
\texttt{cpc}, a caseid value that uniquely identifies each case, the
\texttt{cap} item tells the function to cape the size of the survey
weights at 8 and not allow any case to have a weight larger than that
value. The \texttt{type} item tells the function how it should handle,
if at all, a target population demographic that is very close to the
sample value for that same demographic.

You'll see in the output once you run the \texttt{anesrake} function how
many iterations it took for the raking to converge on this specific set
of weights. Here, it took 15 iterations across the two target
demographic variables.

\textbf{3. Save Weights in Sample Data} Next, we save that newly created
weight as a new variable in our existing sample data, and now we have a
weight variable that we can use in our analysis of the data.

\begin{Shaded}
\begin{Highlighting}[]
\CommentTok{\#Now we save these values as a list and call the list targets}
\CommentTok{\#Step 1: Save the target list }
\NormalTok{targets }\OtherTok{\textless{}{-}} \FunctionTok{list}\NormalTok{(sex, age\_group)}
\CommentTok{\# remember, these names will have to match}
\FunctionTok{names}\NormalTok{(targets) }\OtherTok{\textless{}{-}} \FunctionTok{c}\NormalTok{(}\StringTok{"sex"}\NormalTok{, }\StringTok{"age\_group"}\NormalTok{)}

\CommentTok{\#anesrake(targets, dataframe, caseid, weightvec = NULL, cap = 5,}
\CommentTok{\#verbose = FALSE, maxit = 1000, type = "pctlim", pctlim = 5,}
\CommentTok{\#nlim = 5, filter = 1, choosemethod = "total", iterate = TRUE)}

\CommentTok{\#Step 2 {-} Calculate the Rake Weight}
\FunctionTok{set.seed}\NormalTok{(}\DecValTok{1599385}\NormalTok{) }\CommentTok{\#Set the seed for replication  }
\NormalTok{myweights }\OtherTok{\textless{}{-}} \FunctionTok{anesrake}\NormalTok{(targets, sample, }
                      \AttributeTok{caseid =}\NormalTok{ sample}\SpecialCharTok{$}\NormalTok{caseid, }\AttributeTok{cap =} \DecValTok{8}\NormalTok{, }\AttributeTok{type =} \StringTok{"nolim"}\NormalTok{, }\AttributeTok{pctlim=}\NormalTok{.}\DecValTok{05}\NormalTok{)}
\end{Highlighting}
\end{Shaded}

\begin{verbatim}
[1] "Raking converged in 12 iterations"
\end{verbatim}

\begin{Shaded}
\begin{Highlighting}[]
\CommentTok{\#Step 3 {-} Save the Rake Weight at the end of your sample data}
\NormalTok{sample}\SpecialCharTok{$}\NormalTok{weight  }\OtherTok{\textless{}{-}} \FunctionTok{unlist}\NormalTok{(myweights[}\DecValTok{1}\NormalTok{])}
\end{Highlighting}
\end{Shaded}

\hypertarget{reviewing-the-newly-created-survey-weights}{%
\section{Reviewing the Newly Created Survey
Weights}\label{reviewing-the-newly-created-survey-weights}}

Before we start the analysis of the weighted data, let's examine the
newly created survey weights saved in our sample data.

With only 2 target weighting variables with 10 total categories combined
between them, we can examine the weights individually by group. To do
this, we will use the \texttt{srvyr} package to examine the weight size
by the target groups.

\begin{Shaded}
\begin{Highlighting}[]
\CommentTok{\#Displays summary of the weight size to see range}
\FunctionTok{summary}\NormalTok{(sample}\SpecialCharTok{$}\NormalTok{weight)}
\end{Highlighting}
\end{Shaded}

\begin{verbatim}
   Min. 1st Qu.  Median    Mean 3rd Qu.    Max. 
 0.5347  0.6086  0.6997  1.0000  1.0912  4.8008 
\end{verbatim}

\begin{Shaded}
\begin{Highlighting}[]
\CommentTok{\#Shows the weight size by demographic groups used in the weighting scheme}
\NormalTok{sample }\SpecialCharTok{\%\textgreater{}\%} 
  \FunctionTok{as\_survey}\NormalTok{(}\AttributeTok{weights =} \FunctionTok{c}\NormalTok{(weight)) }\SpecialCharTok{\%\textgreater{}\%}
  \FunctionTok{group\_by}\NormalTok{(sex, age\_group) }\SpecialCharTok{\%\textgreater{}\%} 
  \FunctionTok{summarise}\NormalTok{(}\AttributeTok{weight =} \FunctionTok{survey\_mean}\NormalTok{(weight, }\AttributeTok{na.rm =}\NormalTok{ T))}
\end{Highlighting}
\end{Shaded}

\begin{verbatim}
# A tibble: 10 x 4
# Groups:   sex [2]
   sex        age_group weight weight_se
   <dbl+lbl>  <dbl+lbl>  <dbl>     <dbl>
 1 1 [Female] 1 [18-29]  4.22   0       
 2 1 [Female] 2 [30-39]  1.51   2.86e-17
 3 1 [Female] 3 [40-49]  1.09   0       
 4 1 [Female] 4 [50-64]  0.535  0       
 5 1 [Female] 5 [65+]    0.700  0       
 6 2 [Male]   1 [18-29]  4.80   0       
 7 2 [Male]   2 [30-39]  1.72   3.22e-17
 8 2 [Male]   3 [40-49]  1.24   0       
 9 2 [Male]   4 [50-64]  0.609  0       
10 2 [Male]   5 [65+]    0.796  0       
\end{verbatim}

Now we see the weight size for each of the 10 groups that we weighted
our sample data on. Obviously, with more demographic variables including
in the weighting scheme this list would get much more cumbersome but for
pedagogical purposes it is important to look at these values to
understand their meaning.

For females between the age of 18-29 the weight equals 4.21. This means
that females between the ages of 18-29 are under-represented in the
sample data since the value is over 1. Fundamentally what this means is
that for each female between the ages of 18-29 in the sample data, they
are ``speaking'' for 4.21 females between the age of 18-29 from the
target population. Compare this value to females between the ages of 50
and 64 (group 4) who have a weight value of .53. This means that females
in this age group were over-represented in the sample data since the
weight value is under 1.

We should also look at the weighted demographic values to ensure the
weights worked as we hope they do - i.e.~the weighted sample demographic
values match the target population values.

\begin{Shaded}
\begin{Highlighting}[]
\CommentTok{\#Shows the unweighted size of the age\_group demographics}
\NormalTok{sample }\SpecialCharTok{\%\textgreater{}\%}
  \FunctionTok{group\_by}\NormalTok{(age\_group) }\SpecialCharTok{\%\textgreater{}\%}
  \FunctionTok{summarise}\NormalTok{(}\AttributeTok{n =} \FunctionTok{n}\NormalTok{()) }\SpecialCharTok{\%\textgreater{}\%} 
  \FunctionTok{mutate}\NormalTok{(}\AttributeTok{proportion =}\NormalTok{ n }\SpecialCharTok{/}\FunctionTok{sum}\NormalTok{(n))}
\end{Highlighting}
\end{Shaded}

\begin{verbatim}
# A tibble: 5 x 3
  age_group     n proportion
  <dbl+lbl> <int>      <dbl>
1 1 [18-29]    33     0.0412
2 2 [30-39]   101     0.126 
3 3 [40-49]   118     0.148 
4 4 [50-64]   306     0.382 
5 5 [65+]     242     0.302 
\end{verbatim}

\begin{Shaded}
\begin{Highlighting}[]
\CommentTok{\#Shows the weighted size of the age\_group demographics}
\NormalTok{sample }\SpecialCharTok{\%\textgreater{}\%}
  \FunctionTok{as\_survey}\NormalTok{(}\AttributeTok{weights =} \FunctionTok{c}\NormalTok{(weight)) }\SpecialCharTok{\%\textgreater{}\%}
  \FunctionTok{group\_by}\NormalTok{(age\_group) }\SpecialCharTok{\%\textgreater{}\%}
  \FunctionTok{summarise}\NormalTok{(}\AttributeTok{n =} \FunctionTok{survey\_total}\NormalTok{()) }\SpecialCharTok{\%\textgreater{}\%} 
  \FunctionTok{mutate}\NormalTok{(}\AttributeTok{proportion =}\NormalTok{ n }\SpecialCharTok{/}\FunctionTok{sum}\NormalTok{(n))}
\end{Highlighting}
\end{Shaded}

\begin{verbatim}
# A tibble: 5 x 4
  age_group     n  n_se proportion
  <dbl+lbl> <dbl> <dbl>      <dbl>
1 1 [18-29]  146. 24.9       0.182
2 2 [30-39]  162. 15.2       0.203
3 3 [40-49]  136  11.6       0.17 
4 4 [50-64]  174.  7.87      0.218
5 5 [65+]    182.  9.78      0.227
\end{verbatim}

\begin{Shaded}
\begin{Highlighting}[]
\CommentTok{\#Saves the weighted \& unweighted size of the age\_group demographics}
\NormalTok{ag\_w}\OtherTok{\textless{}{-}}\NormalTok{sample }\SpecialCharTok{\%\textgreater{}\%}
  \FunctionTok{as\_survey}\NormalTok{(}\AttributeTok{weights =} \FunctionTok{c}\NormalTok{(weight)) }\SpecialCharTok{\%\textgreater{}\%}
  \FunctionTok{group\_by}\NormalTok{(age\_group) }\SpecialCharTok{\%\textgreater{}\%}
  \FunctionTok{summarise}\NormalTok{(}\AttributeTok{n =} \FunctionTok{survey\_total}\NormalTok{()) }\SpecialCharTok{\%\textgreater{}\%} 
  \FunctionTok{mutate}\NormalTok{(}\AttributeTok{weighted\_sample =}\NormalTok{ n }\SpecialCharTok{/}\FunctionTok{sum}\NormalTok{(n))}

\NormalTok{ag\_uw}\OtherTok{\textless{}{-}}\NormalTok{ sample }\SpecialCharTok{\%\textgreater{}\%}
  \FunctionTok{group\_by}\NormalTok{(age\_group) }\SpecialCharTok{\%\textgreater{}\%}
  \FunctionTok{summarise}\NormalTok{(}\AttributeTok{n =} \FunctionTok{n}\NormalTok{()) }\SpecialCharTok{\%\textgreater{}\%} 
  \FunctionTok{mutate}\NormalTok{(}\AttributeTok{unweighted\_sample =}\NormalTok{ n }\SpecialCharTok{/}\FunctionTok{sum}\NormalTok{(n))}
\NormalTok{ag\_combo}\OtherTok{\textless{}{-}}\FunctionTok{left\_join}\NormalTok{(ag\_w, ag\_uw, }\AttributeTok{by =} \StringTok{"age\_group"}\NormalTok{, }\AttributeTok{suffix =} \FunctionTok{c}\NormalTok{(}\StringTok{""}\NormalTok{, }\StringTok{"\_pop"}\NormalTok{)) }\SpecialCharTok{\%\textgreater{}\%}
  \FunctionTok{group\_by}\NormalTok{(age\_group)}

\NormalTok{ag\_combo}\SpecialCharTok{$}\NormalTok{ag\_diff\_per}\OtherTok{\textless{}{-}} \DecValTok{100}\SpecialCharTok{*}\NormalTok{(ag\_combo}\SpecialCharTok{$}\NormalTok{weighted\_sample}\SpecialCharTok{{-}}\NormalTok{ag\_combo}\SpecialCharTok{$}\NormalTok{unweighted\_sample)}
\NormalTok{ag\_combo}
\end{Highlighting}
\end{Shaded}

\begin{verbatim}
# A tibble: 5 x 7
# Groups:   age_group [5]
  age_group     n  n_se weighted_sample n_pop unweighted_sample ag_diff_per
  <dbl+lbl> <dbl> <dbl>           <dbl> <int>             <dbl>       <dbl>
1 1 [18-29]  146. 24.9            0.182    33            0.0412       14.1 
2 2 [30-39]  162. 15.2            0.203   101            0.126         7.68
3 3 [40-49]  136  11.6            0.17    118            0.148         2.25
4 4 [50-64]  174.  7.87           0.218   306            0.382       -16.4 
5 5 [65+]    182.  9.78           0.227   242            0.302        -7.55
\end{verbatim}

\begin{Shaded}
\begin{Highlighting}[]
\FunctionTok{print}\NormalTok{(ag\_combo}\SpecialCharTok{$}\NormalTok{weighted\_sample)}
\end{Highlighting}
\end{Shaded}

\begin{verbatim}
[1] 0.182 0.203 0.170 0.218 0.227
\end{verbatim}

\begin{Shaded}
\begin{Highlighting}[]
\FunctionTok{print}\NormalTok{(targets)}
\end{Highlighting}
\end{Shaded}

\begin{verbatim}
$sex
[1] 0.525 0.475

$age_group
[1] 0.182 0.203 0.170 0.218 0.227
\end{verbatim}

We see that the weighted age\_group values match the target population
values we inputted earlier so this weighting scheme seems to be working
in the way that we hoped it would.

\begin{Shaded}
\begin{Highlighting}[]
\CommentTok{\#Shows the unweighted size of the sex demographics}
\NormalTok{sample }\SpecialCharTok{\%\textgreater{}\%}
  \FunctionTok{group\_by}\NormalTok{(sex) }\SpecialCharTok{\%\textgreater{}\%}
  \FunctionTok{summarise}\NormalTok{(}\AttributeTok{n =} \FunctionTok{n}\NormalTok{()) }\SpecialCharTok{\%\textgreater{}\%} 
  \FunctionTok{mutate}\NormalTok{(}\AttributeTok{proportion =}\NormalTok{ n }\SpecialCharTok{/}\FunctionTok{sum}\NormalTok{(n))}
\end{Highlighting}
\end{Shaded}

\begin{verbatim}
# A tibble: 2 x 3
  sex            n proportion
  <dbl+lbl>  <int>      <dbl>
1 1 [Female]   423      0.529
2 2 [Male]     377      0.471
\end{verbatim}

\begin{Shaded}
\begin{Highlighting}[]
\CommentTok{\#Shows the weighted size of the sex demographics}
\NormalTok{sample }\SpecialCharTok{\%\textgreater{}\%}
  \FunctionTok{as\_survey}\NormalTok{(}\AttributeTok{weights =} \FunctionTok{c}\NormalTok{(weight)) }\SpecialCharTok{\%\textgreater{}\%}
  \FunctionTok{group\_by}\NormalTok{(sex) }\SpecialCharTok{\%\textgreater{}\%}
  \FunctionTok{summarise}\NormalTok{(}\AttributeTok{n =} \FunctionTok{survey\_total}\NormalTok{()) }\SpecialCharTok{\%\textgreater{}\%} 
  \FunctionTok{mutate}\NormalTok{(}\AttributeTok{proportion =}\NormalTok{ n }\SpecialCharTok{/}\FunctionTok{sum}\NormalTok{(n))}
\end{Highlighting}
\end{Shaded}

\begin{verbatim}
# A tibble: 2 x 4
  sex            n  n_se proportion
  <dbl+lbl>  <dbl> <dbl>      <dbl>
1 1 [Female]   420  22.0      0.525
2 2 [Male]     380  20.4      0.475
\end{verbatim}

For the \texttt{sex} measure, the unweighted and weighted values matched
nearly identically, but that is because the unweighted sample nearly
matched the target population identically without any statistical
correction. In these instances, it is common to drop the demographic
variable that does not need much adjustment. The standard limit is 5\%
or less should not get an adjustment meaning that you should only apply
weights with the target population values and the sample values are 5\%
points or more different.

\hypertarget{evaluating-influence-of-weights-on-reported-mean-values-in-the-sample-data---smallish-weights}{%
\section{Evaluating Influence of Weights on Reported Mean Values in the
Sample Data - Smallish
Weights}\label{evaluating-influence-of-weights-on-reported-mean-values-in-the-sample-data---smallish-weights}}

Now let's see what impact these weights have on our sample values.
First, we use the \texttt{surveys} package must create a new dataframe
that incorporates the survey weights. Let's call it
\texttt{sample.weighted} to signal that this is the weighted version of
the sample data.

We need to calculate the weighted and unweighted means of the same
variable. Using the \texttt{fracking2} variable which measures support
for fracking in the Colorado which various safety measures, we can
compare the influence of the survey weights on the conclusions we would
draw about support for fracking in Colorado.

Once we run the following R chunk, we see that there is virtually no
difference between the weighted and unweighted estimates of how
supportive Coloradoans are of fracking. Why is this? This occurs
sometimes when the weights that applied simply do not change the sample
composition enough to have an influence on the overall sample mean.

\begin{Shaded}
\begin{Highlighting}[]
\NormalTok{fracking\_uw}\OtherTok{\textless{}{-}}\NormalTok{sample }\SpecialCharTok{\%\textgreater{}\%} \CommentTok{\#Looks at the unweighted support for fracking in CO}
  \FunctionTok{summarise}\NormalTok{(}\AttributeTok{unweight\_support =} \FunctionTok{mean}\NormalTok{(fracking2, }\AttributeTok{na.rm =}\NormalTok{ T))}

\NormalTok{fracking\_w}\OtherTok{\textless{}{-}}\NormalTok{sample }\SpecialCharTok{\%\textgreater{}\%} \CommentTok{\#Looks at the weighted support for fracking in CO}
  \FunctionTok{as\_survey}\NormalTok{(}\AttributeTok{weights =} \FunctionTok{c}\NormalTok{(weight)) }\SpecialCharTok{\%\textgreater{}\%}
   \FunctionTok{summarise}\NormalTok{(}\AttributeTok{weight\_support =} \FunctionTok{survey\_mean}\NormalTok{(fracking2, }\AttributeTok{na.rm =}\NormalTok{ T))}


\NormalTok{fracking\_combo}\OtherTok{\textless{}{-}}\FunctionTok{cbind}\NormalTok{(fracking\_uw, fracking\_w ) }

\NormalTok{fracking\_combo }\OtherTok{\textless{}{-}} \FunctionTok{mutate}\NormalTok{(fracking\_combo, }\AttributeTok{difference =}\NormalTok{ weight\_support }\SpecialCharTok{{-}}\NormalTok{ unweight\_support)}
\NormalTok{fracking\_combo}
\end{Highlighting}
\end{Shaded}

\begin{verbatim}
  unweight_support weight_support weight_support_se  difference
1        0.5277778      0.5025514         0.0229558 -0.02522642
\end{verbatim}

\begin{Shaded}
\begin{Highlighting}[]
\CommentTok{\#Gubernatorial Vote Choice {-} Weighted \& Unweighted }
\NormalTok{gov\_w}\OtherTok{\textless{}{-}}\NormalTok{sample }\SpecialCharTok{\%\textgreater{}\%}
  \FunctionTok{as\_survey}\NormalTok{(}\AttributeTok{weights =} \FunctionTok{c}\NormalTok{(weight)) }\SpecialCharTok{\%\textgreater{}\%}
  \FunctionTok{filter}\NormalTok{(}\SpecialCharTok{!}\FunctionTok{is.na}\NormalTok{(gov\_choice)) }\SpecialCharTok{\%\textgreater{}\%} 
  \FunctionTok{group\_by}\NormalTok{(gov\_choice) }\SpecialCharTok{\%\textgreater{}\%}
  \FunctionTok{summarise}\NormalTok{(}\AttributeTok{n =} \FunctionTok{survey\_total}\NormalTok{()) }\SpecialCharTok{\%\textgreater{}\%} 
  \FunctionTok{mutate}\NormalTok{(}\AttributeTok{weight\_support =}\NormalTok{ n }\SpecialCharTok{/}\FunctionTok{sum}\NormalTok{(n)) }

\NormalTok{gov\_uw}\OtherTok{\textless{}{-}}\NormalTok{sample }\SpecialCharTok{\%\textgreater{}\%}
  \FunctionTok{group\_by}\NormalTok{(gov\_choice) }\SpecialCharTok{\%\textgreater{}\%}
  \FunctionTok{filter}\NormalTok{(}\SpecialCharTok{!}\FunctionTok{is.na}\NormalTok{(gov\_choice)) }\SpecialCharTok{\%\textgreater{}\%} 
  \FunctionTok{summarise}\NormalTok{(}\AttributeTok{n =} \FunctionTok{n}\NormalTok{()) }\SpecialCharTok{\%\textgreater{}\%} 
  \FunctionTok{mutate}\NormalTok{(}\AttributeTok{unweight\_support =}\NormalTok{ n }\SpecialCharTok{/}\FunctionTok{sum}\NormalTok{(n))}

\NormalTok{gov\_combo}\OtherTok{\textless{}{-}}\FunctionTok{cbind}\NormalTok{(gov\_uw, gov\_w) }


\NormalTok{gov\_combo}\SpecialCharTok{$}\NormalTok{diff }\OtherTok{\textless{}{-}}\NormalTok{ gov\_combo}\SpecialCharTok{$}\NormalTok{weight\_support }\SpecialCharTok{{-}}\NormalTok{ gov\_combo}\SpecialCharTok{$}\NormalTok{unweight\_support}
\NormalTok{gov\_combo}
\end{Highlighting}
\end{Shaded}

\begin{verbatim}
  gov_choice   n unweight_support gov_choice         n      n_se weight_support
1          1 413          0.51625          1 438.30863 22.858613     0.54788579
2          2 346          0.43250          2 312.66152 17.843921     0.39082691
3          3  30          0.03750          3  31.10659  6.029889     0.03888324
4          4  11          0.01375          4  17.92325  7.817047     0.02240406
          diff
1  0.031635792
2 -0.041673095
3  0.001383243
4  0.008654059
\end{verbatim}

\hypertarget{create-new-weighting-scheme-that-incorporates-more-demographic-variables}{%
\section{Create New Weighting Scheme That Incorporates More Demographic
Variables}\label{create-new-weighting-scheme-that-incorporates-more-demographic-variables}}

Typically, when creating survey weights you will include more than just
2 demographic variables into your weighting scheme. Here, we use 5
variables to create a new weight: sex, age, race/ethnicity, education,
and partisanship.

\begin{Shaded}
\begin{Highlighting}[]
\CommentTok{\#Save new vectors with target population values for weights }
\NormalTok{sex }\OtherTok{\textless{}{-}} \FunctionTok{c}\NormalTok{(.}\DecValTok{525}\NormalTok{, .}\DecValTok{475}\NormalTok{)  }\DocumentationTok{\#\#Target values for females and males; label order (female, male)}
\NormalTok{age\_group  }\OtherTok{\textless{}{-}} \FunctionTok{c}\NormalTok{(.}\DecValTok{132}\NormalTok{, .}\DecValTok{183}\NormalTok{, .}\DecValTok{15}\NormalTok{, .}\DecValTok{248}\NormalTok{, .}\DecValTok{287}\NormalTok{)   }\CommentTok{\#Target values for 5 age groups }
\NormalTok{race\_4 }\OtherTok{\textless{}{-}}\FunctionTok{c}\NormalTok{(.}\DecValTok{7143}\NormalTok{, .}\DecValTok{0501}\NormalTok{, .}\DecValTok{1768}\NormalTok{, .}\DecValTok{0588}\NormalTok{) }\CommentTok{\#Target values race/ethnic identities {-} white, black, Hispanic, all others}
\NormalTok{educ }\OtherTok{\textless{}{-}}\FunctionTok{c}\NormalTok{(.}\DecValTok{2075}\NormalTok{, .}\DecValTok{2445}\NormalTok{, .}\DecValTok{0828}\NormalTok{, .}\DecValTok{2398}\NormalTok{, .}\DecValTok{2254}\NormalTok{) }\CommentTok{\#Target values education {-} HS or less, Some college, AA, BA, Graduate degree}
\NormalTok{pid\_4 }\OtherTok{\textless{}{-}}\FunctionTok{c}\NormalTok{(.}\DecValTok{3375}\NormalTok{, .}\DecValTok{2838}\NormalTok{, .}\DecValTok{335}\NormalTok{, .}\DecValTok{0437}\NormalTok{) }\CommentTok{\#Target values Party Registration {-} (Democrats, Independents, Republicans, All 3rd Parties)  }

\CommentTok{\#Combine the demographic vectors into a list}
\NormalTok{targets }\OtherTok{\textless{}{-}} \FunctionTok{list}\NormalTok{(sex, age\_group, race\_4, educ, pid\_4)}
\CommentTok{\# remember, these names will have to match the column names \& order in the sample data }
\FunctionTok{names}\NormalTok{(targets) }\OtherTok{\textless{}{-}} \FunctionTok{c}\NormalTok{(}\StringTok{"sex"}\NormalTok{, }\StringTok{"age\_group"}\NormalTok{, }\StringTok{"race\_4"}\NormalTok{, }\StringTok{"educ"}\NormalTok{, }\StringTok{"pid\_4"}\NormalTok{)}

\FunctionTok{set.seed}\NormalTok{(}\DecValTok{1984}\NormalTok{)}
\NormalTok{myweights }\OtherTok{\textless{}{-}} \FunctionTok{anesrake}\NormalTok{(targets, sample, }
                      \AttributeTok{caseid =}\NormalTok{ sample}\SpecialCharTok{$}\NormalTok{caseid, }\AttributeTok{cap =} \DecValTok{8}\NormalTok{, }\AttributeTok{type =} \StringTok{"pctlim"}\NormalTok{, }\AttributeTok{pctlim=}\NormalTok{.}\DecValTok{05}\NormalTok{)    }
\end{Highlighting}
\end{Shaded}

\begin{verbatim}
[1] "Raking converged in 20 iterations"
[1] "Raking converged in 21 iterations"
\end{verbatim}

\begin{Shaded}
\begin{Highlighting}[]
\NormalTok{sample}\SpecialCharTok{$}\NormalTok{full\_weight  }\OtherTok{\textless{}{-}} \FunctionTok{unlist}\NormalTok{(myweights[}\DecValTok{1}\NormalTok{])}

\FunctionTok{summary}\NormalTok{(sample}\SpecialCharTok{$}\NormalTok{full\_weight)}
\end{Highlighting}
\end{Shaded}

\begin{verbatim}
   Min. 1st Qu.  Median    Mean 3rd Qu.    Max. 
 0.3218  0.5685  0.7672  1.0000  1.0729  7.9999 
\end{verbatim}

Let's look at how well the weights worked to match the sample data to
the target population values for the \texttt{race4} and \texttt{educ5}
measures.

\begin{Shaded}
\begin{Highlighting}[]
\CommentTok{\#Shows the weighted size of the educ5 demographics}
\DocumentationTok{\#\#We can also bind the two values together to calculate their differences }
\NormalTok{ed\_w}\OtherTok{\textless{}{-}}\NormalTok{sample }\SpecialCharTok{\%\textgreater{}\%}
  \FunctionTok{as\_survey}\NormalTok{(}\AttributeTok{weights =} \FunctionTok{c}\NormalTok{(full\_weight)) }\SpecialCharTok{\%\textgreater{}\%}
  \FunctionTok{group\_by}\NormalTok{(educ) }\SpecialCharTok{\%\textgreater{}\%}
  \FunctionTok{summarise}\NormalTok{(}\AttributeTok{n =} \FunctionTok{survey\_total}\NormalTok{()) }\SpecialCharTok{\%\textgreater{}\%} 
  \FunctionTok{mutate}\NormalTok{(}\AttributeTok{weighted\_sample =}\NormalTok{ n }\SpecialCharTok{/}\FunctionTok{sum}\NormalTok{(n))}

\NormalTok{ed\_uw}\OtherTok{\textless{}{-}}\NormalTok{ sample }\SpecialCharTok{\%\textgreater{}\%}
  \FunctionTok{group\_by}\NormalTok{(educ) }\SpecialCharTok{\%\textgreater{}\%}
  \FunctionTok{summarise}\NormalTok{(}\AttributeTok{n =} \FunctionTok{n}\NormalTok{()) }\SpecialCharTok{\%\textgreater{}\%} 
  \FunctionTok{mutate}\NormalTok{(}\AttributeTok{unweighted\_sample =}\NormalTok{ n }\SpecialCharTok{/}\FunctionTok{sum}\NormalTok{(n))}
\NormalTok{ed\_combo}\OtherTok{\textless{}{-}}\FunctionTok{left\_join}\NormalTok{(ed\_w, ed\_uw, }\AttributeTok{by =} \StringTok{"educ"}\NormalTok{) }\SpecialCharTok{\%\textgreater{}\%}
  \FunctionTok{group\_by}\NormalTok{(educ)}

\NormalTok{ed\_combo}\SpecialCharTok{$}\NormalTok{ed\_diff\_per}\OtherTok{\textless{}{-}} \DecValTok{100}\SpecialCharTok{*}\NormalTok{(ed\_combo}\SpecialCharTok{$}\NormalTok{weighted\_sample}\SpecialCharTok{{-}}\NormalTok{ed\_combo}\SpecialCharTok{$}\NormalTok{unweighted\_sample)}
\NormalTok{ed\_combo}
\end{Highlighting}
\end{Shaded}

\begin{verbatim}
# A tibble: 5 x 7
# Groups:   educ [5]
  educ                  n.x  n_se weighted_sample   n.y unweighted_sam~1 ed_di~2
  <dbl+lbl>           <dbl> <dbl>           <dbl> <int>            <dbl>   <dbl>
1 1 [HS grad or less] 166.  22.1           0.208     92            0.115   9.25 
2 2 [Some college]    196.  17.2           0.244    187            0.234   1.07 
3 3 [AA degree]        66.2  7.29          0.0828    96            0.12   -3.72 
4 4 [BA]              192.  14.3           0.240    249            0.311  -7.15 
5 5 [Post-grad]       180.  14.3           0.225    176            0.22    0.540
# ... with abbreviated variable names 1: unweighted_sample, 2: ed_diff_per
\end{verbatim}

\begin{Shaded}
\begin{Highlighting}[]
\CommentTok{\#Shows the weighted size of the race4 demographics}
\DocumentationTok{\#\#We can also bind the two values together to calculate their differences }
\NormalTok{r\_w}\OtherTok{\textless{}{-}}\NormalTok{sample }\SpecialCharTok{\%\textgreater{}\%}
  \FunctionTok{as\_survey}\NormalTok{(}\AttributeTok{weights =} \FunctionTok{c}\NormalTok{(full\_weight)) }\SpecialCharTok{\%\textgreater{}\%}
  \FunctionTok{group\_by}\NormalTok{(race\_4) }\SpecialCharTok{\%\textgreater{}\%}
  \FunctionTok{summarise}\NormalTok{(}\AttributeTok{n =} \FunctionTok{survey\_total}\NormalTok{()) }\SpecialCharTok{\%\textgreater{}\%} 
  \FunctionTok{mutate}\NormalTok{(}\AttributeTok{weighted\_sample =}\NormalTok{ n }\SpecialCharTok{/}\FunctionTok{sum}\NormalTok{(n))}

\NormalTok{r\_uw}\OtherTok{\textless{}{-}}\NormalTok{ sample }\SpecialCharTok{\%\textgreater{}\%}
  \FunctionTok{group\_by}\NormalTok{(race\_4) }\SpecialCharTok{\%\textgreater{}\%}
  \FunctionTok{summarise}\NormalTok{(}\AttributeTok{n =} \FunctionTok{n}\NormalTok{()) }\SpecialCharTok{\%\textgreater{}\%} 
  \FunctionTok{mutate}\NormalTok{(}\AttributeTok{unweighted\_sample =}\NormalTok{ n }\SpecialCharTok{/}\FunctionTok{sum}\NormalTok{(n))}
\NormalTok{r\_combo}\OtherTok{\textless{}{-}}\FunctionTok{left\_join}\NormalTok{(r\_w, r\_uw, }\AttributeTok{by =} \StringTok{"race\_4"}\NormalTok{) }\SpecialCharTok{\%\textgreater{}\%}
  \FunctionTok{group\_by}\NormalTok{(race\_4)}

\NormalTok{r\_combo}\SpecialCharTok{$}\NormalTok{r\_diff\_per}\OtherTok{\textless{}{-}} \DecValTok{100}\SpecialCharTok{*}\NormalTok{(r\_combo}\SpecialCharTok{$}\NormalTok{weighted\_sample}\SpecialCharTok{{-}}\NormalTok{r\_combo}\SpecialCharTok{$}\NormalTok{unweighted\_sample)}
\NormalTok{r\_combo}
\end{Highlighting}
\end{Shaded}

\begin{verbatim}
# A tibble: 4 x 7
# Groups:   race_4 [4]
  race_4          n.x  n_se weighted_sample   n.y unweighted_sample r_diff_per
  <dbl+lbl>     <dbl> <dbl>           <dbl> <int>             <dbl>      <dbl>
1 1 [White]     571.  14.0           0.714    692            0.865     -15.1  
2 2 [Black]      40.1  9.97          0.0501    22            0.0275      2.26 
3 3 [Hispanic]  141.  25.1           0.177     41            0.0512     12.6  
4 4 [Other POC]  47.0  7.56          0.0588    45            0.0562      0.255
\end{verbatim}

\begin{Shaded}
\begin{Highlighting}[]
\CommentTok{\#Shows the weighted size of the pid\_4 registered voter variable }
\DocumentationTok{\#\#We can also bind the two values together to calculate their differences }
\NormalTok{pid\_w}\OtherTok{\textless{}{-}}\NormalTok{sample }\SpecialCharTok{\%\textgreater{}\%}
  \FunctionTok{as\_survey}\NormalTok{(}\AttributeTok{weights =} \FunctionTok{c}\NormalTok{(full\_weight)) }\SpecialCharTok{\%\textgreater{}\%}
  \FunctionTok{group\_by}\NormalTok{(pid\_4) }\SpecialCharTok{\%\textgreater{}\%}
  \FunctionTok{summarise}\NormalTok{(}\AttributeTok{n =} \FunctionTok{survey\_total}\NormalTok{()) }\SpecialCharTok{\%\textgreater{}\%} 
  \FunctionTok{mutate}\NormalTok{(}\AttributeTok{weighted\_sample =}\NormalTok{ n }\SpecialCharTok{/}\FunctionTok{sum}\NormalTok{(n))}

\NormalTok{pid\_uw}\OtherTok{\textless{}{-}}\NormalTok{ sample }\SpecialCharTok{\%\textgreater{}\%}
  \FunctionTok{group\_by}\NormalTok{(pid\_4) }\SpecialCharTok{\%\textgreater{}\%}
  \FunctionTok{summarise}\NormalTok{(}\AttributeTok{n =} \FunctionTok{n}\NormalTok{()) }\SpecialCharTok{\%\textgreater{}\%} 
  \FunctionTok{mutate}\NormalTok{(}\AttributeTok{unweighted\_sample =}\NormalTok{ n }\SpecialCharTok{/}\FunctionTok{sum}\NormalTok{(n))}
\NormalTok{pid\_combo}\OtherTok{\textless{}{-}}\FunctionTok{left\_join}\NormalTok{(pid\_w, pid\_uw, }\AttributeTok{by =} \StringTok{"pid\_4"}\NormalTok{) }\SpecialCharTok{\%\textgreater{}\%}
  \FunctionTok{group\_by}\NormalTok{(pid\_4)}

\NormalTok{pid\_combo}\SpecialCharTok{$}\NormalTok{pid\_diff\_per}\OtherTok{\textless{}{-}} \DecValTok{100}\SpecialCharTok{*}\NormalTok{(pid\_combo}\SpecialCharTok{$}\NormalTok{weighted\_sample}\SpecialCharTok{{-}}\NormalTok{pid\_combo}\SpecialCharTok{$}\NormalTok{unweighted\_sample)}
\NormalTok{pid\_combo}
\end{Highlighting}
\end{Shaded}

\begin{verbatim}
# A tibble: 4 x 7
# Groups:   pid_4 [4]
  pid_4             n.x  n_se weighted_sample   n.y unweighted_sample pid_diff~1
  <dbl+lbl>       <dbl> <dbl>           <dbl> <int>             <dbl>      <dbl>
1 1 [Democrat]    270.  21.1           0.338    283            0.354      -1.62 
2 2 [Independent] 227.  17.2           0.284    214            0.268       1.63 
3 3 [Republican]  268.  19.8           0.335    266            0.332       0.250
4 4 [Other]        35.0  6.55          0.0437    37            0.0462     -0.255
# ... with abbreviated variable name 1: pid_diff_per
\end{verbatim}

\begin{Shaded}
\begin{Highlighting}[]
\CommentTok{\#Gubernatorial Vote Choice {-} Weighted \& Unweighted }
\NormalTok{gov\_w}\OtherTok{\textless{}{-}}\NormalTok{sample }\SpecialCharTok{\%\textgreater{}\%}
  \FunctionTok{as\_survey}\NormalTok{(}\AttributeTok{weights =} \FunctionTok{c}\NormalTok{(full\_weight)) }\SpecialCharTok{\%\textgreater{}\%}
  \FunctionTok{filter}\NormalTok{(}\SpecialCharTok{!}\FunctionTok{is.na}\NormalTok{(gov\_choice)) }\SpecialCharTok{\%\textgreater{}\%} 
  \FunctionTok{group\_by}\NormalTok{(gov\_choice) }\SpecialCharTok{\%\textgreater{}\%}
  \FunctionTok{summarise}\NormalTok{(}\AttributeTok{n =} \FunctionTok{survey\_total}\NormalTok{()) }\SpecialCharTok{\%\textgreater{}\%} 
  \FunctionTok{mutate}\NormalTok{(}\AttributeTok{weight\_support =}\NormalTok{ n }\SpecialCharTok{/}\FunctionTok{sum}\NormalTok{(n)) }

\NormalTok{gov\_uw}\OtherTok{\textless{}{-}}\NormalTok{sample }\SpecialCharTok{\%\textgreater{}\%}
  \FunctionTok{group\_by}\NormalTok{(gov\_choice) }\SpecialCharTok{\%\textgreater{}\%}
  \FunctionTok{filter}\NormalTok{(}\SpecialCharTok{!}\FunctionTok{is.na}\NormalTok{(gov\_choice)) }\SpecialCharTok{\%\textgreater{}\%} 
  \FunctionTok{summarise}\NormalTok{(}\AttributeTok{n =} \FunctionTok{n}\NormalTok{()) }\SpecialCharTok{\%\textgreater{}\%} 
  \FunctionTok{mutate}\NormalTok{(}\AttributeTok{unweight\_support =}\NormalTok{ n }\SpecialCharTok{/}\FunctionTok{sum}\NormalTok{(n))}

\NormalTok{gov\_combo}\OtherTok{\textless{}{-}}\FunctionTok{cbind}\NormalTok{(gov\_uw, gov\_w) }


\NormalTok{gov\_combo}\SpecialCharTok{$}\NormalTok{diff }\OtherTok{\textless{}{-}}\NormalTok{ gov\_combo}\SpecialCharTok{$}\NormalTok{weight\_support }\SpecialCharTok{{-}}\NormalTok{ gov\_combo}\SpecialCharTok{$}\NormalTok{unweight\_support}

\NormalTok{gov\_outcome}\OtherTok{\textless{}{-}}\FunctionTok{cbind}\NormalTok{(gov\_combo}\SpecialCharTok{$}\NormalTok{gov\_choice, gov\_combo}\SpecialCharTok{$}\NormalTok{weight\_support, gov\_combo}\SpecialCharTok{$}\NormalTok{unweight\_support, gov\_combo}\SpecialCharTok{$}\NormalTok{diff) }

\FunctionTok{colnames}\NormalTok{(gov\_outcome) }\OtherTok{\textless{}{-}} \FunctionTok{c}\NormalTok{(}\StringTok{"candidate"}\NormalTok{, }\StringTok{"weighted support"}\NormalTok{, }\StringTok{"unewighted support"}\NormalTok{, }\StringTok{"diff"}\NormalTok{) }
\NormalTok{gov\_outcome}
\end{Highlighting}
\end{Shaded}

\begin{verbatim}
     candidate weighted support unewighted support         diff
[1,]         1       0.50298793            0.51625 -0.013262068
[2,]         2       0.43459577            0.43250  0.002095774
[3,]         3       0.04328807            0.03750  0.005788066
[4,]         4       0.01912823            0.01375  0.005378227
\end{verbatim}

\begin{Shaded}
\begin{Highlighting}[]
\DocumentationTok{\#\#\#\#Comparing weighted to unweighted fracking support in Colorado }

\NormalTok{fracking\_uw}\OtherTok{\textless{}{-}}\NormalTok{sample }\SpecialCharTok{\%\textgreater{}\%} \CommentTok{\#Looks at the unweighted support for fracking in CO}
  \FunctionTok{summarise}\NormalTok{(}\AttributeTok{unweight\_support =} \FunctionTok{mean}\NormalTok{(fracking2, }\AttributeTok{na.rm =}\NormalTok{ T))}

\NormalTok{fracking\_w}\OtherTok{\textless{}{-}}\NormalTok{sample }\SpecialCharTok{\%\textgreater{}\%} \CommentTok{\#Looks at the weighted support for fracking in CO}
  \FunctionTok{as\_survey}\NormalTok{(}\AttributeTok{weights =} \FunctionTok{c}\NormalTok{(full\_weight)) }\SpecialCharTok{\%\textgreater{}\%}
   \FunctionTok{summarise}\NormalTok{(}\AttributeTok{weight\_support =} \FunctionTok{survey\_mean}\NormalTok{(fracking2, }\AttributeTok{na.rm =}\NormalTok{ T))}


\NormalTok{fracking\_combo}\OtherTok{\textless{}{-}}\FunctionTok{cbind}\NormalTok{(fracking\_uw, fracking\_w ) }

\NormalTok{fracking\_combo }\OtherTok{\textless{}{-}} \FunctionTok{mutate}\NormalTok{(fracking\_combo, }\AttributeTok{frack\_diff =}\NormalTok{ weight\_support }\SpecialCharTok{{-}}\NormalTok{ unweight\_support)}

\CommentTok{\#Do the same analysis for gun control then pot laws}
\NormalTok{gc\_uw}\OtherTok{\textless{}{-}}\NormalTok{sample }\SpecialCharTok{\%\textgreater{}\%} \CommentTok{\#Looks at the unweighted support for fracking in CO}
  \FunctionTok{summarise}\NormalTok{(}\AttributeTok{unweight\_support =} \FunctionTok{mean}\NormalTok{(gun\_control2, }\AttributeTok{na.rm =}\NormalTok{ T))}

\NormalTok{gc\_w}\OtherTok{\textless{}{-}}\NormalTok{sample }\SpecialCharTok{\%\textgreater{}\%} \CommentTok{\#Looks at the weighted support for fracking in CO}
  \FunctionTok{as\_survey}\NormalTok{(}\AttributeTok{weights =} \FunctionTok{c}\NormalTok{(full\_weight)) }\SpecialCharTok{\%\textgreater{}\%}
   \FunctionTok{summarise}\NormalTok{(}\AttributeTok{weight\_support =} \FunctionTok{survey\_mean}\NormalTok{(gun\_control2, }\AttributeTok{na.rm =}\NormalTok{ T))}


\NormalTok{gc\_combo}\OtherTok{\textless{}{-}}\FunctionTok{cbind}\NormalTok{(gc\_uw, gc\_w ) }

\NormalTok{gc\_combo }\OtherTok{\textless{}{-}} \FunctionTok{mutate}\NormalTok{(gc\_combo, }\AttributeTok{gun\_control\_diff =}\NormalTok{ weight\_support }\SpecialCharTok{{-}}\NormalTok{ unweight\_support)}

\NormalTok{pl\_uw}\OtherTok{\textless{}{-}}\NormalTok{sample }\SpecialCharTok{\%\textgreater{}\%} \CommentTok{\#Looks at the unweighted support for fracking in CO}
  \FunctionTok{summarise}\NormalTok{(}\AttributeTok{unweight\_support =} \FunctionTok{mean}\NormalTok{(pot\_law2, }\AttributeTok{na.rm =}\NormalTok{ T))}

\NormalTok{pl\_w}\OtherTok{\textless{}{-}}\NormalTok{sample }\SpecialCharTok{\%\textgreater{}\%} \CommentTok{\#Looks at the weighted support for fracking in CO}
  \FunctionTok{as\_survey}\NormalTok{(}\AttributeTok{weights =} \FunctionTok{c}\NormalTok{(full\_weight)) }\SpecialCharTok{\%\textgreater{}\%}
   \FunctionTok{summarise}\NormalTok{(}\AttributeTok{weight\_support =} \FunctionTok{survey\_mean}\NormalTok{(pot\_law2, }\AttributeTok{na.rm =}\NormalTok{ T))}


\NormalTok{pl\_combo}\OtherTok{\textless{}{-}}\FunctionTok{cbind}\NormalTok{(pl\_uw, pl\_w ) }

\NormalTok{pl\_combo }\OtherTok{\textless{}{-}} \FunctionTok{mutate}\NormalTok{(pl\_combo, }\AttributeTok{pot\_law\_diff =}\NormalTok{ weight\_support }\SpecialCharTok{{-}}\NormalTok{ unweight\_support)}


\NormalTok{combo}\OtherTok{\textless{}{-}}\FunctionTok{cbind}\NormalTok{(}\DecValTok{100}\SpecialCharTok{*}\NormalTok{(fracking\_combo}\SpecialCharTok{$}\NormalTok{frack\_diff),}\DecValTok{100}\SpecialCharTok{*}\NormalTok{(gc\_combo}\SpecialCharTok{$}\NormalTok{gun\_control\_diff), }\DecValTok{100}\SpecialCharTok{*}\NormalTok{(pl\_combo}\SpecialCharTok{$}\NormalTok{pot\_law\_diff)) }

\FunctionTok{colnames}\NormalTok{(combo) }\OtherTok{\textless{}{-}} \FunctionTok{c}\NormalTok{(}\StringTok{"fracking\_diff"}\NormalTok{, }\StringTok{"gun\_control\_diff"}\NormalTok{, }\StringTok{"pot\_law\_diff"}\NormalTok{) }
\NormalTok{combo}\OtherTok{\textless{}{-}}\FunctionTok{round}\NormalTok{(combo,}\DecValTok{3}\NormalTok{)}
\NormalTok{combo}
\end{Highlighting}
\end{Shaded}

\begin{verbatim}
     fracking_diff gun_control_diff pot_law_diff
[1,]         2.651            0.531       -0.303
\end{verbatim}

Above, shows the impact that the new weighting scheme had on the
differences in support for fracking, new gun control policies, support
for marijuana legalization and a tax revenue proposition on the upcoming
ballot. For fracking, the weighted sample supported fracking 3.3
percentage points higher than the unweighted sample. While that might
not seem like a large difference, in a polarized American electorate 3.3
percentage points easily be the differences between winning an election
or going down in defeat.

The other three items all saw a decrease in support in the weighted
data. Overall, the results showed increased support for a generally
conservative supported issue, fracking, while revealing decreased
support for 3 more liberal supported issues, gun control, marijuana
legalization, and increased governmental spending. This is likely caused
by weighting older voters to be

We can also use the newly created survey weights in regression analyses.
To do so, you first must create a new weighted survey dataset that you
then conduct the analysis on.

\begin{Shaded}
\begin{Highlighting}[]
\CommentTok{\#Using srvyr package to create a new weighted dataset for analysis purposes}
\CommentTok{\#Step 1: Create Weighted Survey Data for Analysis}
\NormalTok{sample.weighted }\OtherTok{\textless{}{-}}\NormalTok{ sample }\SpecialCharTok{\%\textgreater{}\%} 
  \FunctionTok{as\_survey\_design}\NormalTok{(}\AttributeTok{ids =} \DecValTok{1}\NormalTok{, }\CommentTok{\# 1 for no cluster ids; use this for a simple random sample }
                   \AttributeTok{weights =}\NormalTok{ full\_weight, }\CommentTok{\# No weight added}
                   \AttributeTok{strata =} \ConstantTok{NULL} \CommentTok{\# sampling was simple (no strata) }
\NormalTok{                  )}

\NormalTok{nonweighted }\OtherTok{\textless{}{-}}\FunctionTok{lm}\NormalTok{(gambling }\SpecialCharTok{\textasciitilde{}}\NormalTok{ pid\_x }\SpecialCharTok{+}\NormalTok{ ideo5 }\SpecialCharTok{+}\NormalTok{ sex , }\AttributeTok{data=}\NormalTok{sample)}
\NormalTok{weighted }\OtherTok{\textless{}{-}}\FunctionTok{lm}\NormalTok{(gambling }\SpecialCharTok{\textasciitilde{}}\NormalTok{ pid\_x }\SpecialCharTok{+}\NormalTok{ ideo5 }\SpecialCharTok{+}\NormalTok{ sex, }\AttributeTok{data=}\NormalTok{sample, }\AttributeTok{weights=}\NormalTok{sample}\SpecialCharTok{$}\NormalTok{weight)}
\NormalTok{weighted2 }\OtherTok{\textless{}{-}}\FunctionTok{lm}\NormalTok{(gambling }\SpecialCharTok{\textasciitilde{}}\NormalTok{ pid\_x }\SpecialCharTok{+}\NormalTok{ ideo5 }\SpecialCharTok{+}\NormalTok{ sex, }\AttributeTok{data=}\NormalTok{sample, }\AttributeTok{weights=}\NormalTok{sample}\SpecialCharTok{$}\NormalTok{full\_weight)}

\FunctionTok{stargazer}\NormalTok{(nonweighted, weighted,  weighted2, }\AttributeTok{type=}\StringTok{"text"}\NormalTok{)}
\end{Highlighting}
\end{Shaded}

\begin{verbatim}

============================================================
                                    Dependent variable:     
                               -----------------------------
                                         gambling           
                                  (1)       (2)       (3)   
------------------------------------------------------------
pid_x                           -0.017    -0.010    -0.013  
                                (0.031)   (0.032)   (0.029) 
                                                            
ideo5                           -0.059    -0.065   -0.096** 
                                (0.051)   (0.051)   (0.047) 
                                                            
sex                             0.184**  0.215***   0.188** 
                                (0.080)   (0.080)   (0.079) 
                                                            
Constant                       2.391***  2.441***  2.591*** 
                                (0.153)   (0.154)   (0.157) 
                                                            
------------------------------------------------------------
Observations                      669       669       669   
R2                               0.015     0.018     0.025  
Adjusted R2                      0.011     0.013     0.021  
Residual Std. Error (df = 665)   1.030     1.020     0.998  
F Statistic (df = 3; 665)       3.481**  4.032***  5.678*** 
============================================================
Note:                            *p<0.1; **p<0.05; ***p<0.01
\end{verbatim}

Results are largely similar across the regression models, but there are
some slight differences between the unweighted model and the two
weighted ones. This happens because certain respondent's opinions are
being given more or less weight to the overall average relationship
which can cause the conclusions you draw from your analysis to differ.
This is one reason why it is critically important to create your survey
weights using defensible target population values.

\hypertarget{concluding-thoughts}{%
\section{Concluding Thoughts}\label{concluding-thoughts}}

This is an important lesson for the application of the survey weights.
The target population values that you weight your survey sample data to
match can have profound implications on the conclusions you and others
draw from your survey results. In the case, the decision to weight the
survey to give more voice to the Republican members of the sample
influenced the conclusions drawn about support for various policies
being debated in the public realm. This makes it critically important to
make sure that the target population values that are chosen are as
accurate as possible and publicly defensible.

Overall, this tutorial has taken you through how to calculate survey
weights using the \texttt{anesrake} package. Using a sample political
poll, you hopefully learned how to create target demographic population
vectors, which then merge with our sample demographic values. Following
this, you learned how to calculate directly survey weights, evaluate the
success/failure of the survey weighting process, and compare the impact
of using the survey weight on the conclusions drawn from the results.

\begin{Shaded}
\begin{Highlighting}[]
\FunctionTok{sessionInfo}\NormalTok{()}
\end{Highlighting}
\end{Shaded}

\begin{verbatim}
R version 4.1.2 (2021-11-01)
Platform: x86_64-w64-mingw32/x64 (64-bit)
Running under: Windows 10 x64 (build 19042)

Matrix products: default

locale:
[1] LC_COLLATE=English_United Kingdom.1252 
[2] LC_CTYPE=English_United Kingdom.1252   
[3] LC_MONETARY=English_United Kingdom.1252
[4] LC_NUMERIC=C                           
[5] LC_TIME=English_United Kingdom.1252    

attached base packages:
[1] grid      stats     graphics  grDevices utils     datasets  methods  
[8] base     

other attached packages:
 [1] stargazer_5.2.2   data.table_1.14.2 anesrake_0.80     weights_1.0.4    
 [5] Hmisc_4.6-0       Formula_1.2-4     lattice_0.20-45   survey_4.1-1     
 [9] survival_3.2-13   Matrix_1.3-4      srvyr_1.1.2       poliscidata_2.3.0
[13] skimr_2.1.4       forcats_0.5.1     stringr_1.4.0     purrr_0.3.4      
[17] readr_2.1.1       tibble_3.1.6      tidyverse_1.3.1   knitr_1.37       
[21] tidyr_1.1.4       dplyr_1.0.7       ggplot2_3.3.5     haven_2.5.1      

loaded via a namespace (and not attached):
 [1] nlme_3.1-153        bitops_1.0-7        fs_1.5.0           
 [4] lubridate_1.7.10    RColorBrewer_1.1-2  httr_1.4.2         
 [7] repr_1.1.3          tools_4.1.2         backports_1.2.1    
[10] utf8_1.2.2          R6_2.5.1            rpart_4.1.16       
[13] KernSmooth_2.23-20  DBI_1.1.2           colorspace_2.0-2   
[16] nnet_7.3-16         withr_2.4.2         tidyselect_1.1.1   
[19] gridExtra_2.3       compiler_4.1.2      cli_3.0.1          
[22] rvest_1.0.2         htmlTable_2.2.1     mice_3.14.0        
[25] xml2_1.3.3          caTools_1.18.2      scales_1.1.1       
[28] checkmate_2.0.0     digest_0.6.27       minqa_1.2.4        
[31] foreign_0.8-82      rmarkdown_2.11      base64enc_0.1-3    
[34] jpeg_0.1-9          pkgconfig_2.0.3     htmltools_0.5.2    
[37] lme4_1.1-27.1       plotrix_3.8-2       dbplyr_2.1.1       
[40] fastmap_1.1.0       htmlwidgets_1.5.4   rlang_1.0.6        
[43] readxl_1.3.1        rstudioapi_0.13     generics_0.1.0     
[46] jsonlite_1.7.3      gtools_3.9.2        car_3.0-12         
[49] magrittr_2.0.2      Rcpp_1.0.7          munsell_0.5.0      
[52] fansi_0.5.0         abind_1.4-5         lifecycle_1.0.0    
[55] stringi_1.7.4       carData_3.0-4       MASS_7.3-55        
[58] plyr_1.8.6          gplots_3.1.1        gdata_2.18.0.1     
[61] crayon_1.4.1        splines_4.1.2       hms_1.1.1          
[64] pillar_1.8.1        boot_1.3-28         reprex_2.0.1       
[67] glue_1.4.2          evaluate_0.14       mitools_2.4        
[70] latticeExtra_0.6-29 modelr_0.1.8        nloptr_1.2.2.2     
[73] png_0.1-7           vctrs_0.3.8         tzdb_0.1.2         
[76] descr_1.1.5         cellranger_1.1.0    gtable_0.3.0       
[79] assertthat_0.2.1    xfun_0.29           xtable_1.8-4       
[82] broom_0.7.9         cluster_2.1.2       ellipsis_0.3.2     
\end{verbatim}

\bookmarksetup{startatroot}

\hypertarget{measurement-item-scaling}{%
\chapter{Measurement \& Item Scaling}\label{measurement-item-scaling}}

\hypertarget{learning-objectives-6}{%
\section{Learning Objectives}\label{learning-objectives-6}}

\hypertarget{measurement}{%
\section{Measurement}\label{measurement}}

\hypertarget{item-scaling}{%
\section{Item Scaling}\label{item-scaling}}

\bookmarksetup{startatroot}

\hypertarget{hypothesis-testing-analysis-reporting}{%
\chapter{Hypothesis Testing, Analysis, \&
Reporting}\label{hypothesis-testing-analysis-reporting}}

\hypertarget{learning-objectives-7}{%
\section{Learning Objectives}\label{learning-objectives-7}}

\hypertarget{hypothesis-testing}{%
\section{Hypothesis Testing}\label{hypothesis-testing}}

\hypertarget{analysis}{%
\section{Analysis}\label{analysis}}

\hypertarget{reporting}{%
\section{Reporting}\label{reporting}}

\bookmarksetup{startatroot}

\hypertarget{conclusion}{%
\chapter{Conclusion}\label{conclusion}}

\hypertarget{benefits-of-survey-design-analysis-in-r}{%
\section{Benefits of Survey Design \& Analysis in
R}\label{benefits-of-survey-design-analysis-in-r}}

\bookmarksetup{startatroot}

\hypertarget{references}{%
\chapter*{References}\label{references}}
\addcontentsline{toc}{chapter}{References}

\hypertarget{refs}{}
\begin{CSLReferences}{0}{0}
\end{CSLReferences}



\end{document}
